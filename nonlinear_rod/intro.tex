%Введение и постановка задачи 
The mitral valve, also known as the bicuspid valve or left atrioventricular
valve, is a valve with two flaps in the heart, that lies between the left atrium
and the left ventricle. The mitral valve and the tricuspid valve are known
collectively as the atrioventricular valves because they lie between the atria
and the ventricles of the heart.\par
\begin{figure}[H]
  \centering
  \includegraphics[width=0.4\columnwidth]{./fig/mt.png}
  \caption{Mitral valve structure}
  \label{fig:MT}
\end{figure}
Mitral valve has cyclic working conditions.
%Example of such you can see on figure \ref{fig:workMT}.
In normal conditions, blood flows through an open mitral valve during diastole
with contraction of the left atrium, and the mitral valve closes during systole
with contraction of the left ventricle. The valve opens and closes because of
pressure differences, opening when there is greater pressure in the left atrium
than ventricle, and closing when there is greater pressure in the ventricle than
atrium. In abnormal conditions, blood may flow backwards through the valve
(mitral regurgitation) or the mitral valve may be narrowed (mitral stenosis).
Rheumatic heart disease often affects the mitral valve; the valve may also
prolapse with age, and be affected by infective endocarditis.
\begin{figure}[H]\label{fig:workingMT}
  \centering
  \begin{subfigure}[H]{0.45\columnwidth}\label{fig:openMT}
    \includegraphics[width=\columnwidth]{./fig/openMT.png}
    \caption{Open state}
  \end{subfigure}
  \begin{subfigure}[H]{0.4\columnwidth}\label{fig:closedMT}
    \includegraphics[width=\columnwidth]{./fig/closedMT.png}
    \caption{Closed state}
  \end{subfigure}
  \caption{Mitral valve working cycle}
\end{figure}
Mitral valve prolapse (MVP) is a valvular heart disease characterized by the
displacement of an abnormally thickened mitral valve leaflet into the left
atrium during systole. By other words, it is a condition in which the two flaps
of the mitral valve doesn't close smoothly and evenly, but instead bulge
(prolapse) upward into the left atrium.\cite{Hayek2005a}
\begin{figure}[H]\label{fig:compareMT}
  \centering
  \begin{subfigure}[H]{0.42\columnwidth}\label{fig:normalMT}
    \includegraphics[width=\columnwidth]{./fig/normalMT.png}
    \caption{Normal}
  \end{subfigure}
  \begin{subfigure}[H]{0.4\columnwidth}\label{fig:prolapseMT}
    \includegraphics[width=\columnwidth]{./fig/prolapseMT.png}
    \caption{Prolapse}
  \end{subfigure}
  \caption{Comparison of the normal valve and prolapse}
\end{figure}
Providing the surgeon with an anatomically and biomechanically accurate
computional model of a particular patient's mitral heart valve could enable
preoperative surgical planning and potentially improve surgical outcome.\par
