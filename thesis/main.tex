\documentclass[11pt, oneside, a4paper]{report}

\usepackage[utf8]{inputenc}
\usepackage[english]{babel}
\edef\restoreparindent{\parindent=\the\parindent\relax}
\usepackage{parskip}
\restoreparindent
\usepackage{indentfirst}
\setlength{\topmargin}{-1in}
\setlength{\textheight}{10.2in}
\setlength{\textwidth}{7.1in} 
\setlength{\footskip}{0.6in}
\setlength{\oddsidemargin}{-0.3in}
\pretolerance=150
\newcommand{\cm}[1]{\par\textcolor{red}{\textbf{{#1}}}\par}
\newcommand{\noNumChapter}[1]{\chapter*{{#1}}\addcontentsline{toc}{chapter}{{#1}}}
\newcommand{\noNumSection}[1]{\section*{{#1}}\addcontentsline{toc}{section}{{#1}}}
\newcommand{\noNumSubSection}[1]{\subsection*{{#1}}\addcontentsline{toc}{subsection}{{#1}}}
\usepackage[backend=bibtex, natbib=true, style=numeric, sorting=none]{biblatex}
\addbibresource{../library.bib}

\usepackage{csquotes}
\usepackage[svgnames]{xcolor}
\usepackage[colorlinks=true, citecolor=Black, linkcolor=Black, urlcolor=Black]{hyperref}
\usepackage{bookmark}
\usepackage{mathtools}
\usepackage{float}
\usepackage{subcaption}
\usepackage{svg}
\usepackage{algorithm2e}
\usepackage{hyperref}
\begin{document}
\title{Numerical Model of Dynamic Deformation of Fibrous Structural Membranes}

\author{Oleksandr Hubanov\\ 
\textit{Supervisor} \\ prof. habil. dr. Rimantas Kačianauskas
\\
\\
Vilnius Gediminas Technical University}
\maketitle
\tableofcontents

\noNumChapter{INTRODUCTION}%5-6pages
\noNumSection{Problem Formulation}
\noNumSection{Relevance of the Thesis}
\noNumSection{The Object of Research}
\noNumSection{The Aim of the Thesis}
\noNumSection{The Tasks of the Thesis}
\noNumSection{Research Methodology}
\noNumSection{Scientific Novelty of the Thesis}
\noNumSection{Practical Value of the Research Findings}
\noNumSection{The Defended Statements}
\noNumSection{Approval of the Research Findings}
\noNumSection{Structure of the Dissertation}
\chapter{LITERATURE REVIEW}%20 pages
\section{Mitral Valve as Human Body Part}
\subsection{Functional Anatomy of Heart and Mitral Valve}
\subsection{Mitral Valve Function and Dysfunction}
\subsection{Mitral Valve Surgical Repair Techniques}
\section{Computational Models}
\subsection{Discrete Element Method}
\subsection{Fluid-Structure Interaction Models}
\section{Modelling of Biological Soft Tissues}
\subsection{Soft Fibers Structural Models}
\subsection{Mitral Valve and Blood}
\subsection{Computational Models for Mitral Valve Repair}
\subsection{Conclusions and Thesis Tasks Formulation}
\chapter{Discrete Models of Soft Fibers}%10-15 pages
\section{Problem Formulation}
\section{Basic Equations}
\section{Computational Forces}
\subsection{Geometrical nonlinearity}
\subsection{Nonlinear Elasticity}
\subsection{Nonconservative transversial loading}
\subsection{Fibers in Fluid}
\subsection{Integration Methods}
\section{Membrane as A system of Soft Fibers}
\chapter{Application of discrete models}%30 pages
\section{Pulling of Chordae Tendineae}
\section{Chordae Tendineae - Membrane Junction}
\chapter{Developing method of preparing medical data}%20 pages
\section{Patient-Specific Mitral Valve Geometry}
\subsection{Echocardiographic Data Acquisition}
\subsection{Echocardiographic Image Segmentation}
\section{Patient-Specific Geometry Reconstruction}
\subsection{Simplyfiyng Patient-Specific 3D Model To MassPoints}
\chapter{Conclusions}
\noNumChapter{INTRODUCTION}

%\printbibliography[title=List of literature]
\end{document}
