\section*{Integration schemes}
Euler first order explisit
$v(t+\Delta t) = v(t) + a(t) * \Delta t $
$x(t+\Delta t) = x(t) + v(t) * \Delta t $
\par
Runge-Kutta schemes find application within discrete element simulations. While being of high
accuracy they are computationally expensive.
$kr_1 = v(t), kv_1 = a(t)$
$r_2 = r(t) + \frac{1}{2} * \Delta t * kr_1, v_2 = v(t) + \frac{1}{2} * \Delta t * kv_1$
$kr_2 = v_2, kv_2 = a(t + \Delta t)$
$r_3 = r(t) + \frac{1}{2} * \Delta t * kr_2, v_3 = v(t) + \frac{1}{2} * \Delta t * kv_2$
$kr_3 = v_3, kv_3 = a(t + 2\Delta t)$
$r_4 = r(t) + \frac{1}{2} * \Delta t * kr_3, v_4 = v(t) + \frac{1}{2} * \Delta t * kv_3$
$kr_4 = v_4, kv_3 = a(t + 3\Delta t)$
$r(t+\Delta t) = r(t) + \Delta t * (\frac{1}{6} kr_1 + \frac{1}{3} kr_2 + \frac{1}{3} kr_3 + \frac{1}{6} kr_4)$
$v(t+\Delta t) = v(t) + \Delta t * (\frac{1}{6} kv_1 + \frac{1}{3} kv_2 + \frac{1}{3} kv_3 + \frac{1}{6} kv_4)$
\par
The central difference scheme, also known as velocity Verlet is a very widely used integration
method of second order. This was initially proposed by Cundall and Strack (1979) and adopted by
several other authors. Velocities in the upcoming time step t +t/2 and positions at t +t are
calculated as
$v(t + \Delta t/2) = v(t + \Delta t/2) +a(t - \Delta t/2) * \Delta t$
$x(t + \Delta t) = x(t) +v(t + \Delta t/2) * \Delta t$
\par
predictor–corrector schemes very commonly used for molecular dynamics and discrete element
applications are Gear’s schemes. They are based on three stages, whereas in addition to the
predictor and corrector step known from the Adams-method an evaluation step is added. Schemes
considered here are the third order Gear’s method (GPC3) and the fourth order Gear’s method (GPC4).
In the prediction step positions and their higher derivatives are calculated based on Taylor series
expansions as
$c(t+\Delta t, p) = c(t)$
$b(t+\Delta t, p) = b(t) + c(t) *\Delta t$
$a(t+\Delta t, p) = a(t) + b(t) *\Delta t^2 + \frac{1}{6} * c(t) *\Delta t^2$
$v(t+\Delta t, p) = v(t) + a(t) *\Delta t + \frac{1}{2} * b(t) *\Delta t^2 + \frac{1}{6} * c(t) *\Delta t^3$
$x(t+\Delta t, p) = x(t) + v(t) * \Delta t + \frac{1}{2} * a(t) *\Delta t^2 + \frac{1}{6} * b(t) *\Delta t^3 + \frac{1}{24} * c(t) *\Delta t^4$
with the first and second derivative of the accelerations calculated as
for GPC3:
$b(t) = \frac{\mathrm d a(t)}{\mathrm d t}$
$c(t) = 0$
for GPC4:
$b(t) = \frac{\mathrm d a(t)}{\mathrm d t}$
$c(t) = \frac{\mathrm d b(t)}{\mathrm d t}$
\par
In the evaluation step the difference in the accelerations calculated based on the acceleration
$a(t+ \Delta t, p)$ and the acceleration $a(t+\Delta t)$ calculated from positions $x(t+\Delta t,
p)$ and velocities $v(t+\Delta t, p)$ is obtained by
$\Delta a = a(t+\Delta t) − a(t+\Delta t, p)$
\par
In the following, correction step positions and their higher derivatives are calculated based on
their values from the previous time step and the obtained difference in acceleration as
$x(t+\Delta t) = x(t+\Delta t, p) + k1 * \Delta a * \Delta t^2$
$v(t+\Delta t) = v(t+\Delta t, p) + k2 * \Delta a * \Delta t$
$a(t+\Delta t) = a(t+\Delta t, p) + k3 * \Delta a$
$b(t+\Delta t) = b(t+\Delta t, p) + k4 * \frac{\Delta a}{\Delta t}$
$c(t+\Delta t) = c(t+\Delta t, p) + k5 * \frac{\Delta a}{\Delta t^2}$
\par
Gear’s scheme parameters k1-k5
for GPC3:
$k_1 = 1/12, k_2 = 5/12, k_3 = 1, k_4 = 1, k_5 = 0$
for GPC4:
$k_1 = 19/240, k_2 = 3/8, k_3 = 1, k_4 = 3/2, k_5 = 1$
\par
\newpage