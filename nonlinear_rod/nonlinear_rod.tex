% Example LaTeX document for GP111 - note % sign indicates a comment
\documentclass[12pt]{article}
% Default margins are too wide all the way around.  I reset them here
\setlength{\topmargin}{-1.4in}
\setlength{\textheight}{10.8in}
\setlength{\oddsidemargin}{.125in}
\setlength{\textwidth}{6.25in} 
\usepackage[svgnames]{xcolor}
\usepackage{mathtools}
\usepackage{svg}
\svgpath{./fig/}
\usepackage[colorlinks=true, linkcolor=Black, urlcolor=Black]{hyperref}
\begin{document}
\title{Forces in rod}
\author{Oleksandr Hubanov\\
Vilnius Gediminas Technical University}
%\renewcommand{\today}{November 2, 1994}
\maketitle
%{\em like this}. curve font
%\bf bold font
%\rm roman font
\section {Linear case}
Object of work is system of 1D rods. Object has been divided to discrete elements n1, n2 and n3.
All elements are connected to each other through nodes n2, n3 and to special points through n1 and n4(figure 1).
Mat model of discrete system is expressed by equations of motion for nodes. As elements is 1D, nodes will be 1D as well. \par
\textcolor{red}{Discribe global coord system in which object acting} \par
The motion of nodes can be expressed by Newton's equation of motion. If only the normal component of the translational
motion is considered, the equation reduces to\par
\begin{equation}\label{eqn:motionEq}
   F(x)-m\ddot{x}=0
\end{equation}
where $F(x)$ is axial force, $m$ – mass of node and $\ddot{x}$ is acceleration, initial conditions are: $x(0)=0$ and $\dot{x}(0)=V_0$\par
Mass of each node can be calculated, like sum of half mass of each element, which acting in node.\par
\begin{figure}
  \centering
  \includesvg{nodeExtract}    
  \caption{Extracting node from system}    
\end{figure}
For investigating motion of system need to integrate \eqref{eqn:motionEq}. For this propose need to express all possible to act forces in node for each node.
Also need to translate forces which acting in element to node \eqref{eqn:sumF}. 
\begin{equation}\label{eqn:sumF}
   \sum F=F + df
\end{equation}

\pagebreak

\section {Nonlinear case}

\end{document}
\end{verbatim}
\end{document}
