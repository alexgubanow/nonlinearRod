\subsection*{Research aim}
%Цель- сравнить одну стержень и систему
%Хорда описана ломаной линией из элементов первого порядка, 
Mechanic equivalent of fibrous tissue could be described as system of 1D rods.
Example of such flat 2D system are shown on figure \ref{fig:nodeExtract}.
\begin{figure}[H]
  \centering
  \includesvg{systemAtworld}    
  \caption{1D Rod system in global coordinate system}\label{fig:rodSystem}      
\end{figure} 
System consists of discrete elements $e_1$, $e_2$ and $e_3$. All elements are
connected to each other through nodes $n_2$, $n_3$ and to special points through
$n_1$ and $n_4$. Each element $e_n$ of system has own orientation in global
coordinate system. 
From schematic representation of node comes that all vector variables of node should be calculated
 in global coordinate system and element variables in own local coordinate system(figure
 \ref{fig:nodeExtract}). For transformation between coordinate systems direction cosine matrix
 (DCM)\eqref{eqn:DCM} can be used.
\begin{equation}\label{eqn:DCM}
  DCM= \begin{bmatrix}
    cos(X,x)&cos(X,y)&cos(X,z)\\
    cos(Y,x)&cos(Y,y)&cos(Y,z)\\
    cos(Z,x)&cos(Z,y)&cos(Z,z)\\
   \end{bmatrix} 
\end{equation}
where $\{X, Y, Z\}$ is global coordinate system and $\{x,y,z\}$ is local coordinate
system.\par According to primitive scheme of node \ref{fig:nodeExtract}, mass of
each node can be calculated, like sum of half mass of each element, which acting
in node. $m_n=\sum_{e}m_e/2$\par
In case that node does not have external interrupt, like pressure or other
 applied force, schematic represent of node can be as on figure
 \ref{fig:nodeExtract}.\par
\begin{figure}[H]
  \centering
  \includesvg{nodeExtract}    
  \caption{Extracted node from system}\label{fig:nodeExtract}
\end{figure}
Mathematical model of discrete system is expressed by equations of nodes motion.
As elements is 1D, nodes will be 1D as well. All system acting in global
coordinate system $\{X, Y, Z\}$ and each element acting in own local coordinate
system $\{x,y,z\}$.