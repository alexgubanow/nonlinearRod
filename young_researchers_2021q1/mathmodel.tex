\section*{Mathematical model}
The motion of chordae as motion of physical object can be expressed by Newton's
equation of motion:
\begin{equation}\label{eqn:motionEq}
  m\vec{\ddot{x}}= \vec{F}
\end{equation}
where $F$ is internal force of object, $m$ – mass of object and $\ddot{x}$ is
acceleration, $k$ – stiffness coefficient of object, $x$ is displacement of
object and $c$ – damping coefficient of object, $\dot{x}$ is velosity of object.
\begin{figure}[H]\label{fig:chordaeInCoords}
  \centering
  \includesvg[width=0.8\columnwidth]{fig/chordaeInCoords.svg}
  \caption{Chordae tindae in global coordinate system}
\end{figure}
Initial conditions are: $x(0)=0$ and $\dot{x}(0)=V_0$.
Acceleration could be known from \eqref{eqn:motionEq} as:
\begin{equation}\label{eqn:Accel}
  \vec{\ddot{x}}=\vec{F}/m
\end{equation}
Than Euler's scheme of integration can be described like:
\begin{equation}\label{eqn:Velos}
  \vec{\dot{x}}(t +\Delta t)=\vec{\dot{x}}(t)+\vec{\ddot{x}}(t)\Delta t
\end{equation}
\begin{equation}\label{eqn:Displ}
  \vec{x}(t +\Delta t)=\vec{x}(t)+\vec{\dot{x}}(t)\Delta t
\end{equation}\par
\subsection*{Linear deformation}
Internal object force becomes from physical deformation of object it self. In
linear case of study, deformation of element much less compare to element
dimensions. It is expressed by linear geometry
equation\eqref{eqn:linDeformation}, which showing relation between initial
length of element and length in $\Delta t$ state.
\begin{equation}\label{eqn:linDeformation}
  \varepsilon=\frac{dU}{dx}=\frac{l(\Delta t)-l(t)}{l(t)}
\end{equation}
According to Hook law $\sigma=\varepsilon E$ and linear geometry equation
\eqref{eqn:linDeformation}, inner force can be described as:
\begin{equation}\label{eqn:linNxFull}
  \begin{split}
    N(x) &= \int\limits_{Al} \varepsilon EdldA=EA\int\limits_l \varepsilon dl \\
    &=\frac{EA}{l(t)}*(l(t + \Delta t)-l(t))
  \end{split}
\end{equation}
where $l(\Delta t)$ is current length of chordae, $l(t)$ length of chordae at
previuous time moment, $E$ – Young’s modulus for chordae, $A$ – volume of
chordae. To be able to integrate equation of motion, need to express deformation
in equation \eqref{eqn:linNxFull} by differences between displacements of nodes,
to which chordae is connected:
\begin{equation}\label{eqn:linNxWdispl}
  N(x)=\frac{EA}{l(t)}*(x_{i}(t)-x_{j}(t))
\end{equation}\par

%Нелинейное деформ стержня, по времени
%Описать допущение что модуль упрогости = нелинейность
\subsection*{Nonlinear deformation}
Nonlinearity in main mean that chordae can get huge deformation compare to
its demesions. Equation of $F_{elem}$ in this case  would change to
nonlinear form:
\begin{equation}\label{eqn:nonlinNx}
  N(x)= \int\limits_t\int\limits_A \sigma dAdt
\end{equation}
From this equantion comes that cross sectional area and stiffness coefficient
will get nonlinearity.
\par
Changing of cross sectional area over time of chordae is changing its length
over time. Linear geometry equation\eqref{eqn:linDeformation}, showing linear
relations between length, because difference in $\Delta t$ state takes according
initial length of chordae. In case of huge deformation need to recalculate
length of chordae on each time step and take difference of displacement
according to previous time step. In end of geometry equation become to nonlinear
form:
\begin{equation}\label{eqn:nonlinDeformation}
  \varepsilon=\frac{dU}{dx}=\frac{l(\Delta t)-l(t-\Delta t)}{l(\Delta t)}
\end{equation}
Inner force\eqref{eqn:linNxFull} also become to nonlinear form:
\begin{equation}\label{eqn:nonlinNxFull}
  \begin{split}
    N(x) &=  \int\limits_t\int\limits_A \varepsilon EdA=EA\int\limits_t\varepsilon \\
    &=\frac{EA}{l(\Delta t)}*(l(\Delta t)-l(t-\Delta t))
  \end{split}
\end{equation}
where $l$ is current length of element, $l_0$ length of element at $t=0$, $E$ –
Young’s modulus for element material.

And nonlinear equantion of inner force for integration:
\begin{equation}\label{eqn:nonlinNxWdispl}
  N(x)=\frac{EA}{l_0}*(U_{i}-U_{j})
\end{equation}\par
