%Введение и постановка задачи 

\section*{Itroduction}
Review of an integration schemes for nonlinear forces been done in
\cite{Danby2013}, By investigating the characteristics of both the non-linear
elastic and the non-linear damped Hertzian contact models, it has been found
that higher orders of accuracy are recoverable and depends on the degree of the
governing non-linear equation. The numerical errors of the linear and non-linear
force models are however markedly different in character.

Symplectic Euler, Forward Euler, Backward Differenced, Runge–Kutta,
Adams–Bashforth been compared \cite{IACOBELLIS2019111373} The recommended solver
for any large scale system would be the Symplectic Euler scheme due to its
ability to produce near sec- ond order results, its general high accuracy,
simple implementation and low storage requirements.

The numerical errors in idealised discrete element method (DEM) simulations are
reviewed in \cite{Hanley2017} The truncation error has a superlinear
relationship with the simulation time-step, $\Delta t$. Reducing $\Delta t$
substantially reduces $\epsilon $; however, increasing the number of time-steps
required in a simulation inevitably increases the round-off error. This implies
that there is an optimal, implementation- specific choice of $\Delta t$, which
minimises the total error.

Efficacy of contact dynamics (CD) in a variety of indeterminate problems, including
some involving multiple materials, non-spherical shapes, and nonlinear contact
constitutive laws been shown in \cite{Olsen2018a}.

A brief overview is given on the capabilities and on the current limitations of
the Discrete Element Method (DEM) coupled with Computational Fluid Mechanics
(CFD) to simulate chemical reacting moving granular material.\cite{Scherer2017}

Impact of particle shape is inspected in \cite{Zhang2020} Simulations show that
sediment particle’s shape effect on its motion is more obvious in laminar flows
rather than in turbulent flows.

Dynamic stability of beam structures for 8, 16 and 32 layers been investigated
at \cite{Smoljanovic2020} 

Among Euler, semi-implicit Euler, Runge-Kutta and Leapfrog, we found that                                                                                                   
simulation with Leapfrog numerical integration characterizes a mass-spring-damper
system best in terms of the energy loss of the overall system. \cite{Ozguz2011}                                                                                      