\chapter*{Simulation of tissue movement} 
Investigation behavior of biological tissue can be done by estimating movement
of it. Tissues of biological origin, usually had a weak structure and little
stiffness.
\cm{opisat podrobno strukturu i phiz svoystva, sunut kartinki}
\par
Such physical structure imposes restrictions on the possible tools in the
measurement. Another possible problem is the limited number of measurement
attempts. Such limitations of physical measurements call into question the
possibility of such activities in general. An alternative to physical
experiments is the numerical simulation of these experiments. There are a large
number of different numerical modeling methods. All of them are derivatives of
Finite Element Method(FEM) and Discrete Element Method(DEM).
\cm{raspisat chto takoe FEM and DEM in basic way, just to show difference of}
\par
Based on the properties of the material of biological tissues, the most
appropriate method is Mass - Spring modeling(MSM). This method based on ideas of
DEM and basic element here is very know in mechanic simple 1D beam.
\cm{have to show so pic about MSM, with explanation what is going on}
Computational complexity of MSM is much less compare to FEM-based methods,
because of less number of equations to integrating. This important advantage and
physics way have method describes basic element gave to MSM very wide using in
computer games for calculating reality-looks hair or cloth movement in real
time. Also compare to FEM-based methods, MSM gave possibility to make parallel
calculating of each time step. For some real example, such calculation could be
done on Nvidia GPU, by using. \cite{Rasmusson2008} \cite{Amorim2012}
\par
Object of work is system of 1D rods. System consists of discrete elements n1, n2
and n3. All elements are connected to each other through nodes n2, n3 and to
special points through n1 and n4 (figure \ref{fig:rodSystem}). Mathematical
model of discrete system is expressed by equations of motion for nodes. As
elements is 1D, nodes will be 1D as well. All system acting in global coordinate
system $\{X, Y, Z\}$ and each element acting in own local coordinate system
$\{x,y,z\}$ (figure \ref{fig:nodeExtract}).\par
\begin{figure}[ht]
  \centering
  \includesvg{systemAtworld}    
  \caption{1D Rod system in global coordinate system}\label{fig:rodSystem}      
\end{figure}
Let's try to describe minimal possible way to get simulation of such structure as
 shown on figure \ref{fig:rodSystem}. First of all need to understand size of 
In case that node does not have external interrupt, like pressure or other
applied force, schematic represent of node can be as on figure
\ref{fig:nodeExtract}.\par
\begin{figure}[ht]
  \centering
  \includesvg{nodeExtract}    
  \caption{Extracting node from system}\label{fig:nodeExtract}
\end{figure}
From schematic representation of node comes that all vector variables of node
should be calculated in global coordinate system and element variables in own
local coordinate system(figure \ref{fig:nodeExtract}). For transformation between
coordinate systems direction cosine matrix (DCM)\eqref{eqn:DCM} can be used.
\begin{equation}\label{eqn:DCM}
  DCM= \begin{bmatrix}
    cos(X,x)&cos(X,y)&cos(X,z)\\
    cos(Y,x)&cos(Y,y)&cos(Y,z)\\
    cos(Z,x)&cos(Z,y)&cos(Z,z)\\
   \end{bmatrix} 
\end{equation}
where {X, Y, Z} is global coordinate system and {x, y, z} is local coordinate
system.\par According to primitive scheme of node \ref{fig:nodeExtract}, mass of
each node can be calculated, like sum of half mass of each element, which acting
in node. $m_n=\sum_{e}m_e/2$\par