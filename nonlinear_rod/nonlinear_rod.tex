\documentclass[12pt]{report}
\setlength{\topmargin}{-1.1in}
\setlength{\textheight}{10in}
\setlength{\oddsidemargin}{.125in}
\setlength{\textwidth}{6.25in} 
\usepackage[svgnames]{xcolor}
\usepackage[colorlinks=true, linkcolor=Black, urlcolor=Black]{hyperref}
\usepackage{bookmark}
\usepackage{mathtools}
\usepackage{float}
\usepackage{svg}
\svgpath{./fig/}
\begin{document}
\title{Forces in rod}
\author{Oleksandr Hubanov\\
Vilnius Gediminas Technical University}
%\renewcommand{\today}{November 2, 1994}
\maketitle
%{\em like this}. curve font
%\bf bold font
%\rm roman font
\pagebreak
\chapter {1D rod system}
Object of work is system of 1D rods. Object has been divided to discrete elements n1, n2 and n3.
All elements are connected to each other through nodes n2, n3 and to special points through n1 and n4
(figure \ref{fig:rodSystem}). Mathematical model of discrete system is expressed by equations of motion for nodes.
As elements is 1D, nodes will be 1D as well. All system acting in global coordinate system $\{X, Y, Z\}$.\par
\par
\begin{figure}[ht]\label{fig:rodSystem}
  \centering
  \includesvg{systemAtworld}    
  \caption{1D Rod system in global coordinate system}      
\end{figure}
\section{Linear case}
The motion of nodes can be expressed by Newton's equation of motion. If only the normal component of the translational
motion is considered, the equation reduces to\par
\begin{equation}\label{eqn:motionEq}
   F(x)-m\ddot{x}=0
\end{equation}
where $F(x)$ is axial force, $m$ – mass of node and $\ddot{x}$ is acceleration, initial conditions are: $x(0)=0$ and
$\dot{x}(0)=V_0$.\par
Mass of each node can be calculated, like sum of half mass of each element, which acting in node.
$m_n=\sum_{e}m_e/2$
\par
\begin{figure}[ht]\label{fig:nodeExtract}
  \centering
  \includesvg{nodeExtract}    
  \caption{Extracting node from system}    
\end{figure}
All vector varibles of node should be calculated in global coordinate system and of element in own local coordinate 
system(figure \ref{fig:nodeExtract}). For trasformation between coordinate systems direction cosine matrix
(DCM)\eqref{eqn:DCM} can be used.
\begin{equation}\label{eqn:DCM}
  DCM= \begin{bmatrix}
    cos(Xx)&cos(Xy)&cos(Xz)\\
    cos(Yx)&cos(Yy)&cos(Yz)\\
    cos(Zx)&cos(Zy)&cos(Zz)\\
   \end{bmatrix} 
\end{equation}
 where {X, Y, Z} is global coordinate system and {x, y, z} is local coordinate system.\par
For investigating motion of system need to integrate \eqref{eqn:motionEq}. For this propose need to express all possible 
to act forces in node\eqref{eqn:sumF} for each node in relation to their application place. 
\begin{equation}\label{eqn:sumF}
   \overrightarrow{F_n}=\sum\overrightarrow{F}=
   \overrightarrow{F_{ext}}+
   \overrightarrow{F_{elem}}\times[DCM]+
   \overrightarrow{F_{press}}\times[DCM]
\end{equation}\par
$F_{ext}$ is external load force, applied to node in global coordinates. Value of this force for each time step is loaded
from list of loads.\par
$F_{elem}$ is sum of internal forces of each elemen, which acting in node. From each element becomes only half of force to 
node, other half going to neigourh node. In case of 1D element system, internal force of each element can be express like
axial force and it is equal to integral of stress over area:
\begin{equation}\label{eqn:Nx}
  N(x)= \int\limits_A \sigma dA
\end{equation}
For 1D rod system $F_{elem}$ can be expressed like:
\begin{equation}\label{eqn:Felem}
  F_{elem}= \sum_{e}N_e(x)/2
\end{equation}
$F_{press}$ is external pressure, can be described like force applied to element in local coordinates. Value of this force
for each time step is loaded from list of loads. \par
Element force becomes from physical deformation of element. Geometry equation\eqref{eqn:deformation} describing relation between initial length of
element and lenght in $\Delta t$ state, by other words its describing deformation of element over time.
\begin{equation}\label{eqn:deformation}
  \varepsilon=\frac{dU}{dx}=\frac{l-l_0}{l_0}
\end{equation}
According to Hook law $\sigma=\varepsilon E$ and geometry equation \eqref{eqn:deformation}, inner force can be changed to
\begin{equation}\label{eqn:NxFull}
  N(x)= \int\limits_A \varepsilon EdA=EA\int \varepsilon=\frac{EA}{l_0}*(l-l_0)
\end{equation}
where $l$ current length of element, $l_0$ length of element at $t=0$, $E$ – Young’s modulus for element material.
To be able to integrate equation of motion, need to express deformation in equation \eqref{eqn:NxFull} by differenses 
between displacements of nodes, to which element is connected:
\begin{equation}\label{eqn:NxWdispl}
  N(x)=\frac{EA}{l_0}*(U_{i}-U_{j})
\end{equation}\par
Equations of motion for integrating can be described like:
\begin{equation}\label{eqn:Accel}
  \ddot{U}(\Delta t)=Fn/m
\end{equation}
\begin{equation}\label{eqn:Velos}
  \dot{U}(\Delta t)=\dot{U}(t)+\ddot{U}(\Delta t)\Delta t
\end{equation}
\begin{equation}\label{eqn:Displ}
  U(\Delta t)=U(t)+\dot{U}(\Delta t)\Delta t
\end{equation}

\pagebreak

\chapter {Nonlinear case}

\end{document}
